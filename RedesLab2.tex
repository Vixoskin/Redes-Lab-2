\documentclass{article}
\usepackage[utf8]{inputenc}
\usepackage{graphicx}

\title{Lab 2 Redes}
\author{Vicente Lopez\\Esteban León\\Sebastian Antón\\Jorge Ramirez\\Profesor: Jose Alejandro Perez\\Ayudante: Alexis Inzunza}
\date{Abril 2017}

\usepackage{natbib}
\usepackage{graphicx}

\begin{document}
\begin{figure}[h]
\includegraphics[width=0.45\textwidth]{logo_udp.png}
\maketitle
\end{figure}

\section{Indice}
2 Actividades del laboratiorio---------------------------------------------------------------1\\
2.1 Construccion de cable directo-----------------------------------------------------------2\\
2.2 Construccion de cable cruzado---------------------------------------------------------2\\
3 Describa brevemente las categorías existentes de cable UTP y sus usos-------2\\
4 ¿Para qué situaciones debería utilizar un cable STP?-------------------------------3\\
5 Conclusion--------------------------------------------------------------------------------------3\\
6 Bibliografia-------------------------------------------------------------------------------------3\\
\section{Actividades:}
En ambas actividades presentadas a continuación se realizarán de la siguiente manera:\\\\
a) Quitar 5 cm. de la envoltura de ambos extremos del cable.\\
b) Organice los pares de acuerdo a la norma que utilizará\\
c) Aplane y enderece los pares, córtelos en la punta para que queden del mismo largo y rectos\\
d) Inserte los hilos en un cabezal RJ45, asegúrese de que los hilos entren completamente\\
f) Inserte el cabezal RJ45 en el alicate para crimpear el extremo, apretándolo firmemente\\
g) Repita los mismos pasos para el otro extremo del cable y asegúrese de seguir la misma norma\\\\\\

\subsection{Contrucción de cable directo:}
Esta actividad consiste en armar un cable directo, es decir, que cada extremo tendrá la misma norma, la posibles combinación posibles son T568A - T568A o T568B - T568B\\

\subsection{Contrucción de cable cruzado:}
En este caso, la actividad consiste en hacer un cable cruzado, es decir, que cada extremo tendrá una norma diferente, las posible combinaciones posibles serán T568A - T568B o T568B - T568A\\

\section{Describa brevemente las categorías existentes de cable UTP y sus usos:}

Categoría 1: Este tipo de cable es utilizado para las redes telefónicas (no es adecuado para la transferencia de datos en red)\\\\

Categoría 2: Es un tipo de cable no protegido (sin cobertura de aluminio para interferencias), capaz de transmitir datos hasta 4 Mbps. Se usó en las redes ARCnet y Token Ring, pero dejó de ser usado, ya que no es adecuado para la transferencia de datos en red(al igual que el cable de categoría 1).\\\\

Categoría 3: Es un par trenzado que puede transmitir datos hasta 10 Mbps, fue un popular cableado entre los administradores de redes a comienzos de los noventa. Fué reemplazado por el cable categoría 5.\\\\

Categoría 4: Es un par trenzado sin blindar, capaz de transmitir datos hasta 16 Mbps. Es más confiable para la transmisión de datos que el cable de categoría 3. Fue usado en redes token rings, 10BASE-T, 100BASE-T4. Fué reemplazado por el cable de categoría 5e.\\\\

Categoría 5: Cable de par trenzado que transmite datos hasta 100 Mbps (FastEthernet), se usa en redes como ethernet y para llevar señales de servicios básicos de telefonía, token ring y ATM . Estos cables se usan comúnmente para las conexiones de área local.\\\\

Categoría 5e: Es una versión mejorada del cable de categoría 5 tiene una velocidad máxima de transferencia de datos de hasta 1000 Mbps (Gigabit Ethernet), aunque no está soportado oficialmente. Es una excelente opción para la red 1000BASE T.\\\\

Categoría 6: Es una propuesta de par trenzado que es capaz de soportar una transmisión datos de hasta 1 Gb. Es adecuado para las redes 1000BASE T, 100BASE T y 10BASE T. Posee características para evitar difonía y ruido.\\\\

\section{¿Para qué situaciones debería utilizar un cable STP?}
El cable STP (Shielded Twisted Pair o Par trenzado apantallado) es ideal para lugares que tengan una alta y continua transferencia de datos, como una sala de servidores, o una sala de telecomunicaciones,  ya que con sus pares cubiertos de aluminio, evita las interferencias que puedan afectar los procesos de transferencia de datos, evitando la pérdida de estos y la continuidad del servicio. \\\\

\section{Conclusion}
Como conclusión, pudimos conocer en detalle las características de un cable categoría 5 y características generales de las distintas categorías que existen de estos. También vivimos la experiencia de armar un cable categoría 5 (directo y cruzado) con nuestras propias manos dándonos cuenta de las diferencias entre un cable cruzado de uno directo.

\section{Bibliografia}
https://es.wikipedia.org/\\
http://www.siemon.com/es/

\end{document}